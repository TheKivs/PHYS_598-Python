\documentclass[aps,prl,groupedaddress,amsmath,amssymb,nofootinbib,11pt]{revtex4-1}
\pdfoutput=1
\oddsidemargin -40pt
 \evensidemargin -40pt
\hyphenpenalty=10000
\usepackage[
top    = 2.1cm,
bottom = 2 cm,
left   = 1.9cm,
right  = 1.9cm]{geometry}
\usepackage[utf8]{inputenc}

\def\thesection{\Roman{section}}

%\usepackage{mathrsfs}
\usepackage[usenames, dvipsnames]{color}
\usepackage[normalem]{ulem}
\usepackage{amsmath}
\usepackage{caption}
\usepackage{amssymb}
\usepackage{amscd}
\usepackage{enumerate}
\usepackage{amsfonts}
\usepackage{slashed}
\usepackage[colorlinks=true,linkcolor=ForestGreen, citecolor=ForestGreen,urlcolor=ForestGreen]{hyperref}

\usepackage{etoolbox}

%\usepackage[notref,notcite]{showkeys}




%\usepackage[notref,notcite]{showkeys}
%\usepackage{showkeys}





\usepackage{graphicx}
\usepackage{bm}


\linespread{1.196}


\definecolor{davecolor}{rgb}{0.95,  0.5,  0.2}
\def\DV#1{{\color{davecolor}{\bf #1}}}

\def\del{\partial}

\def\eg{{\it e.g.}}
\def\ie{{\it i.e.}}
\def\cf{{\it c.f. }}
\def\etal{{\it et. al.}}



\def\({\left(}
\def\){\right)}

\def\<{\langle}
\def\>{\rangle}
%\def\half{{1\over 2}}

\def\CA{{\cal A}}
\def\CC{{\cal C}}
\def\CD{{\cal D}}
\def\CE{{\cal E}}
\def\CF{{\cal F}}
\def\CG{{\cal G}}
\def\CT{{\cal T}}
\def\CM{{\cal M}}
\def\CN{{\cal N}}
\def\CO{{\cal O}}%AEL
\def\CP{{\cal P}}
\def\CL{{\cal L}}
\def\CV{{\cal V}}
\def\CS{{\cal S}}
\def\CW{{\cal W}}
\def\CX{{\cal X}}%AEL
%\def\bra#1{{\langle}#1|}
%\def\ket#1{|#1\rangle}
\def\bbra#1{{\langle\langle}#1|}
\def\kket#1{|#1\rangle\rangle}
%\def\vev#1{\langle{#1}\rangle}
\def\Dslash{\rlap{\hskip0.2em/}D}
\def\CDslash{\rlap{\hskip0.2em/}{\CD}}

\def\DDslash{{\cal L}}
%\def\Dslash{\rlap{\hskip0.2em/}D}
%\def\vev#1{\langle#1 \rangle}
%\def\CO{{\cal O}}
%\def\half{{1\over 2}}


\newcommand{\cit}[1]{  [{\bf\texttt{#1}}]}

\makeatletter
\newcommand*\bigcdot{\mathpalette\bigcdot@{.5}}
\newcommand*\bigcdot@[2]{\mathbin{\vcenter{\hbox{\scalebox{#2}{$\m@th#1\bullet$}}}}}
\makeatother



\def\Tr{\mathop{\rm Tr}}
\def\tr{\mathop{\rm tr}}

\newcommand\half{{\ensuremath{\frac{1}{2}}}}
\newcommand\p{\ensuremath{\partial}}
\newcommand\evalat[2]{\ensuremath{\left.{#1}\right|_{#2}}}
\newcommand\abs[1]{\ensuremath{\left\lvert{#1}\right\rvert}}
\newcommand\no[1]{{{:}{#1}{:}}}
\newcommand\transpose{{\ensuremath{\text{\sf T}}}}
\newcommand\field[1]{{\ensuremath{\mathbb{{#1}}}}}
\newcommand\order[1]{{\ensuremath{{\mathcal O}({#1})}}}
\newcommand\vev[1]{{\ensuremath{\left\langle{#1}\right\rangle}}}
\newcommand\anti[2]{\ensuremath{\bigl\{{#1},{#2}\bigr\}}}
\newcommand\com[2]{\ensuremath{\bigl[{#1},{#2}\bigr]}}
\newcommand\ket[1]{\ensuremath{\lvert{#1}\rangle}}
\newcommand\bra[1]{\ensuremath{\langle{#1}\rvert}}
\newcommand\lie[2]{\ensuremath{\pounds_{{#1}} {#2}}}
%\newcommand\sfrac[2]{\ensuremath{{({#1})}/{({#2})}}} % or
\newcommand\sfrac[2]{\ensuremath{\frac{#1}{#2}}}
\newcommand\lvec[2][]{\ensuremath{\overleftarrow{{#2}_{#1}}}}
\newcommand\rvec[2][]{\ensuremath{\overrightarrow{{#2}_{#1}}}}
%\newcommand\lvec[2][]{\ensuremath{\vec{{#2}}_{#1}}}
%\newcommand\rvec[2][]{\ensuremath{\stackrel{\rightarrow}{{#2}}_{#1}}}


\newcommand{\myfig}[3]{
	\begin{figure}[ht]
	\centering
	\includegraphics[width=#2cm]{figs/#1}\caption{#3}\label{fig:#1}
	\end{figure}
	}
\newcommand{\littlefig}[2]{
	\includegraphics[width=#2cm]{figs/#1}
	}


\newcommand\speceq{\ensuremath{\stackrel{\star}{=}}}
\newcommand\conj[1]{{\ensuremath{\left({#1}\right)^*}}}



%\newcommand{\field}[1]{\ensuremath{\mathbb{#1}}}
\newcommand{\A}{\field{A}}
%\newcommand{\CC}{\field{C}}
\newcommand{\DD}{\field{D}}
\newcommand{\J}{\field{J}}
\newcommand{\GG}{\field{G}}
\newcommand{\FF}{\field{F}}
\newcommand{\QQ}{\field{Q}}
\newcommand{\HH}{\field{H}}
\newcommand{\LL}{\field{L}}
\newcommand{\KK}{\field{K}}
\newcommand{\NN}{\field{N}}
\newcommand{\PP}{\field{P}}
\newcommand{\RR}{\field{R}}
\newcommand{\TT}{\field{T}}
\newcommand{\ZZ}{\field{Z}}





\newcommand{\be}{\begin{equation*}}
\newcommand{\ee}{\end{equation*}}
\newcommand{\bea}{\begin{eqnarray}}
\newcommand{\eea}{\end{eqnarray}}
\newcommand{\bwt}{\begin{widetext}}
\newcommand{\ewt}{\end{widetext}}
\newcommand{\nn}{\nonumber\\}
\newcommand{\bi}{\begin{itemize}}
\newcommand{\ei}{\end{itemize}}
\newcommand{\ben}{\begin{enumerate}}
\newcommand{\een}{\end{enumerate}}
\newcommand{\bca}{\begin{cases}}
\newcommand{\eca}{\end{cases}}
\newcommand{\bln}{\begin{align}}
\newcommand{\eln}{\end{align}}
\newcommand{\bst}{\begin{split}}
\newcommand{\est}{\end{split}}

\newcommand\al{{\alpha}}
\newcommand\ep{\epsilon}
\newcommand\sig{\sigma}
\newcommand\Sig{\Sigma}
\newcommand\lam{\lambda}
\newcommand\Lam{\Lambda}
\newcommand\om{\omega}
\newcommand\Om{\Omega}
\newcommand\vt{\vartheta}
\newcommand\ga{{\ensuremath{{\gamma}}}}
\newcommand\Ga{{\ensuremath{{\Gamma}}}}
\newcommand\de{{\ensuremath{{\delta}}}}
\newcommand\De{{\ensuremath{{\Delta}}}}
\newcommand\vp{\varphi}
\newcommand\ze{{\zeta}}

\newcommand\da{{\dagger}}
\newcommand\nab{{\nabla}}
\newcommand\Th{{\Theta}}
\def\th{{\theta}}

\newcommand\ra{{\rightarrow}}
\newcommand\Lra{{\Longrightarrow}}
\newcommand\ov{\over}
\newcommand\ha{{\half}}
\newcommand\papr{{2 \pi \apr}}
\newcommand\apr{{\ensuremath{{\alpha'}}}}
\def\le{\left}
\def\ri{\right}

\newcommand\sA{{\ensuremath{{\mathcal A}}}}
\newcommand\sB{{\ensuremath{{\mathcal B}}}}
\newcommand\sC{{\ensuremath{{\mathcal C}}}}
\newcommand\sD{{\ensuremath{{\mathcal D}}}}
\newcommand\sF{{\ensuremath{{\mathcal F}}}}
\newcommand\sI{{\ensuremath{{\mathcal I}}}}
\newcommand\sG{{\ensuremath{{\mathcal G}}}}
\newcommand\sK{{\ensuremath{{\mathcal K}}}}
\newcommand\sH{{\ensuremath{{\mathcal H}}}}
\newcommand\sL{{\ensuremath{{\mathcal L}}}}
\newcommand\sM{{\ensuremath{{\mathcal M}}}}
\newcommand\sN{{\ensuremath{{\mathcal N}}}}
\newcommand\sO{{\ensuremath{{\mathcal O}}}}
\newcommand\sR{{\ensuremath{{\mathcal R}}}}
\newcommand\sZ{{\ensuremath{{\mathcal Z}}}}
\newcommand\sn{{\ensuremath{{\mathfrak n}}}}


\newcommand\sg{g}

\newcommand\sV{{\mathcal V}}
\newcommand\sJ{{\mathcal J}}
\newcommand\sS{{\mathcal S}}

\newcommand\bQ{{\bf Q}}
\newcommand\bT{{\bf T}}
\newcommand\bA{{\bf A}}
\newcommand\bB{{\bf B}}
\newcommand\bC{{\bf C}}
\newcommand\bR{{\bf R}}
\newcommand\bX{{\bf X}}
\newcommand\bY{{\bf Y}}
\newcommand\bI{{\bf I}}
\newcommand\bv{{\bf v}}
\newcommand\bw{{\bf w}}
\newcommand\bII{{\bf II}}
\newcommand\bth{{\boldsymbol \theta}}
\newcommand\bom{{\boldsymbol \omega}}
\newcommand\bq{{\bf Q}_B}
\newcommand\bfb{{\bf b}_{-1}}

\newcommand\bc{{\bar c}}
\newcommand\bb{{\bar b}}

\newcommand\bpsi{{\bar \psi}}


\renewcommand{\Im}{\textrm{Im}\,}
\renewcommand{\Re}{\textrm{Re}\,}


\newcommand{\rd}{\ell_2}
\newcommand{\hmq}{\widehat \mu_q}
\newcommand{\bone}{{\bf 1}}
\newcommand{\vk}{{\vec k}}

\newcommand\tpi{{\tilde \pi}}

\def\tildem{\tilde m}
%\def\psis{\textswab{Y}}
\def\yandz{\Phi}
\def\psis{\yandz}

\newcommand\uz{{\underline{z}}}
\newcommand\utau{{\underline{\tau}}}
\newcommand\ut{{\underline{t}}}
\newcommand\ur{{\underline{r}}}
\newcommand\ui{{\underline{i}}}
\newcommand\uj{{\underline{j}}}
\newcommand\umu{{\underline{\mu}}}
\newcommand\uy{{\underline{y}}}

\newcommand{\dbyd}[1]{\ensuremath{\left(\frac{\partial}{\partial #1}\right)}}
\newcommand{\itoj}{\ensuremath{i\leftrightarrow j}}
\newcommand{\ux}{\underline x}


\newcommand{\uM}{\underline M}
\newcommand{\cA}{\mathcal{A}}
\newcommand{\cB}{\mathcal{B}}
\newcommand{\cC}{\mathcal{C}}
\newcommand{\cD}{\mathcal{D}}

%\def\Dslash{\rlap{\hskip0.2em/}D}
\def\Aslash{\rlap{\hskip0.2em/}A}
\def\pslash{\rlap{\hskip0.1em/}p}
\def\kslash{\rlap{\hskip0.1em/}k}

\newcommand\psinorm{\boldsymbol{\psi}}
\newcommand\Psinorm{\boldsymbol{\Psi}}
\newcommand\Phinorm{\boldsymbol{\Phi}}
\def\vertexZ{\Lambda}
%\def\dk{\(d\vec k\)}
\def\Q{\mathcal{Q}}

\newcommand\Psinon{\mathfrak{Y}}



\begin{document}


\section{PHYS 598 SDA RECITATION 6 \& 7 - PROBLEM SET}

**Starred problems to be done at the end. An indispensable reference is 
\texttt{Problems and Solutions for Groups, Lie Groups, Lie Algebras With Applications, Steeb et al.}**\\

\textbf{I. Practice with Lie Groups:} In mathematics, a \textbf{Lie group} (pronounced ``Lee") is a group that is also a \emph{differentiable manifold}, with the property that the group operations are compatible with the smooth structure. The standard Wiki-level example are 2$\times$2 real invertible matrices under multiplication, denoted by GL($2, \mathbb{R})$. This group is an open subset of $\mathbb{R}^4$, and is disconnected - it has two connected components corresponding to the positive and negative values of the determinant.

\emph{If these statements are unclear, please ask me before proceeding!} For a more formal treatment, consult Ch.2 of \href{http://www.springer.com/cda/content/document/cda_downloaddocument/9781461442431-c1.pdf?SGWID=0-0-45-1339003-p174506084}{\texttt{Homogeneous Finsler Spaces, Springer Monographs in Mathematics}}.\\


\textbf{I-1.} A complex $2n \times 2n$ matrix $S$ is called \emph{symplectic} if $S^tJS$ = $J$, where
\begin{equation*}
J = \begin{pmatrix}
\mathbb{O}_n&\mathbb{I}_n\\
-\mathbb{I}_n&\mathbb{O}_n
\end{pmatrix}
\end{equation*}
and $\mathbb{I}_n$ is the $n \times n$ identity matrix, etc. Prove that the set of $2n \times 2n$ complex symplectic matrices, denoted by Sp$(n, \mathbb{C})$, is a matrix Lie group\footnote{Some people denote the group of $2n \times 2n$ complex symplectic matrices by Sp($2n, \mathbb C$). But I don't like those people.} [i.e., it is a topologically closed\footnote{Topologically, not necessarily smooth, as defined above!} subgroup of GL($2n, \mathbb{C}$)]. \emph{To show this, first prove closure, then show inclusion of the identity and finally, the existence of an inverse.}\\

\textbf{I-2.} The Lie group SU$(1 , 1)$ is defined as the group of $2\times 2$ matrices $V$ that satisfy
\[V\sigma_3V^\dagger = \sigma_3~~~ \text{and}~~~ \text{det } V = 1.\] The Lie group SO$(2 , 1)$ is the group of transformations on vectors $\vec{x} \in \mathbb{R}^3$ (with determinant =1) that preserves $x_1^2+x_2^2-x_3^3$. Display the \emph{homomorphism} from SU$(1 , 1)$ onto SO$(2 , 1)$. \emph{All that is required is to show that the group multiplication law is preserved.}\\

\textbf{I-3.$^*$} Let $\alpha,\beta \in \mathbb R$. Let $T$ be the $2\times 2$ unitary diagonal matrix
\begin{equation*}
T(\alpha,\beta) = \begin{pmatrix}e^{i\alpha}&0\\0&e^{i\beta}\end{pmatrix}.
\end{equation*}
Find the condition on a unitary $2 \times 2$ matrix $U$ s.t. $UT(\alpha,\beta)U^{-1} = T(\alpha',\beta')$. \emph{Think of all possible cases}.\\

\textbf{II. Homogenous spaces:} A smooth manifold $M$ endowed with a transitive, smooth action by a Lie group $G$ is called a Homogeneous $G$-space. \emph{If any of these terms are unfamiliar to you, open to }$\S2$ \href{https://www.mathi.uni-heidelberg.de/~lee/MenelaosSS16.pdf}{\texttt{here}}. Your task is to understand the examples of Isometry groups in the Wiki article on \href{https://en.wikipedia.org/wiki/Homogeneous_space#Examples}{\texttt{Homogeneous spaces}}.
\pagebreak

\textbf{III. A Permutation group}... is a group $G$ whose elements are permutations of a given set $M$ and whose group operation is the composition of permutations in $G$ (which are thought of as bijective functions from the set $M$ to itself). Since permutations are bijections of a set, they can be represented by Cauchy's two-line notation. This notation lists each of the elements of $M$ in the first row, and for each element, its image under the permutation below it in the second row. If $\sigma$ is a permutation of the set $ M=\{x_{1},x_{2},\ldots ,x_{n}\}$ then,
\[\sigma = \begin{pmatrix}x_1&x_2&x_3&\ldots&x_n\\ \sigma(x_1)&\sigma(x_2)&\sigma(x_3)&\ldots&\sigma(x_n)\end{pmatrix}.\]

\textbf{III-1.} The group of all permutations of a set $M$ is the symmetric group Sym($M$). Often, the notation $S_n$ is interchangeably used (since set $M$ has $n$ elements). The term \emph{permutation group} thus means a subgroup of the symmetric group. \emph{Convince yourself of this last statement.}\\

\textbf{III-2.} Consider the permutation group $S_3$, such that
\[S_3 = \{\sigma_1 = (),~ \sigma_2 = (12), ~\sigma_3=(13),~ \sigma_4 = (23), ~\sigma_5 = (123), ~\sigma_6 = (132)\}.\]
Compute/look up the \emph{Cayley table} of $S_3$. \emph{Convince yourself that $S_3$ is isomorphic to the dihedral group $D_3$. This is very relevant to students working in crystallography - and you might just know it already!} \\

\textbf{III-3.$^*$} Show that the inverse of a permutation matrix is its transpose.\\

\textbf{IV. Reading exercise:} We shall want to understand  Lie groups and Homogeneous spaces in the context of \emph{Grassmannian manifolds}. Our aim is to get through as much as possible of $\S2$ in this \href{https://www.ime.usp.br/~piccione/Downloads/NotasXIEscola.pdf}{\texttt{PDF}}.\\

\textbf{V.$^*$ Further readings:} The link \href{https://icerm.brown.edu/materials/Slides/sp-s13-off_weeks/Schubert_varieties_and_Schubert_calculus_]_Sara_Billey,_University_of_Washington.pdf}{\texttt{here}} explains \emph{Schubert varieties} very simply. It's worth a read!
\end{document}
